\documentclass{CRPITStyle} 
%\usepackage{epsfig}   % Packages to use if you wish
%\usepackage{lscape}   % 
\usepackage{harvard}  
\pagestyle{empty}
\thispagestyle{empty}
\hyphenation{roddick}

\begin{document}

\title{ Conferences in Research and Practice in Information Technology - Style Guide}
\author{John F. Roddick 
\and 
Anne Other}
\affiliation{School of Informatics and Engineering \\
Flinders University of South Australia, \\
PO Box 2100, Adelaide, South Australia 5001, \\
Email:~{\tt roddick@cs.flinders.edu.au}}

\maketitle

%Include either the full copyright statement:
%\toappear{Copyright \copyright 2003, Australian Computer Society, Inc.  This paper appeared at the 2nd Australian Institute of Computer Ethics Conference (AICE2000), Canberra.  Conferences in Research and Practice in Information Technology, Vol. 1. J. Weckert, Ed. Reproduction for academic, not-for profit purposes permitted provided this text is included. }
%or more easily (and recommended) use the alternative:

\newcommand\conferencenameandplace{Twenty-Sixth Australasian Computer Science Conference (ACSC2003), Adelaide, Australia}
\newcommand\volumenumber{16}
\newcommand\conferenceyear{2003}
\newcommand\editorname{Michael Oudshoorn}
\toappearstandard 

\toappear{You can use the \slash toappear again to add a second footnote for grant and other information.}

\begin{abstract}
This paper describes the manner in which papers should be formatted for papers adhering to the ACS series, Conference in Research and Practice in Information Technology.  The abstract should be a maximum of 250 words and should clearly identify the content of the paper.  
\end{abstract}
\vspace{.1in}

\noindent {\em Keywords:} As required..

\section{Introduction}
Normal text is 10 point, Times Roman in two columns.  Page size is A4 with 0.8in borders on all sides, two columns with 0.2 in between them.  There should not be a blank line between paragraphs.  If the style is being used properly, a natural looking gap will be produced as \LaTeX thinks fit.

Headings should use the heading styles as shown.  Numbering is automatic.

\section{Heading Level 1}
\subsection{Heading Level 2}
\subsubsection{Heading Level 3}
Headings below level 3 should be avoided.
Tables and figures should ideally be confined to one column but where this is not possible should be located at the top of a page.  Each be given a caption using the caption style, For example, 

\begin{figure}[htb]
\fbox{\parbox[b]{.99\linewidth}{
\vskip 0.5cm
\centerline{Figure Content}
\vskip 0.5cm}}
\caption{\protect\label{xyz}  Caption}
\end{figure}

Citations should use the author date format only.  For example, 

...as proved by Snodgrass (1987) \nocite{Snodgrass87} and Fayyed {\it et al.} \cite{FPS96} and referred to in other works \cite{BenZvi82,Bentley86,AIS93} the process..., etc.

For further information regarding formats, please contact the editor-in-chief.

\section{Submitting Camera Ready copy}
Once your paper has been amended as required by the refereeing process, the source should be uploaded to the specified ftp server as described on the CRPIT website.  An e-mail will be sent to the first of the email addresses you list above confirming receipt once it has been received and printed.  
Note that only PDF and postscript files are accepted as upload formats.  Do remember that CRPIT papers are A4, not US letter size (thus on the dvips you often have to specify {\tt -t a4}) -- see website.

\section{References}
References should be in the standard JRPIT format, examples of which are shown below (shown is a conference paper, a thesis, a book, a book section and a journal paper in that order).  The required formats can be obtained by including the {\it Harvard} package and using style {\it agsm}.  The files {\tt harvard.sty} and {\tt agsm.bst} are available at the CRPIT website.

\bibliographystyle{agsm}    % or some other suitable package.

%\bibliography{CRPITExample}  % often included from a separate file.

\begin{thebibliography}{xx}

\harvarditem[Agrawal et~al.]{Agrawal, Imielinski \harvardand\
  Swami}{1993}{AIS93}
Agrawal, R., Imielinski, T. \harvardand\ Swami, A.  \harvardyearleft
  1993\harvardyearright , Mining association rules between sets of items in
  large databases, {\em in} `ACM SIGMOD International Conference on Management
  of Data', Vol.~22, ACM Press, Washington DC, USA, pp.~207--216.

\harvarditem{Ben-Zvi}{1982}{BenZvi82}
Ben-Zvi, J.  \harvardyearleft 1982\harvardyearright , The time relational
  model, Ph.D., University of California, Los Angeles.

\harvarditem{Bentley}{1986}{Bentley86}
Bentley, J.  \harvardyearleft 1986\harvardyearright , {\em Programming pearls},
  Addison-Wesley.

\harvarditem[Fayyad et~al.]{Fayyad, Piatetsky-Shapiro \harvardand\
  Smyth}{1996}{FPS96}
Fayyad, U.~M., Piatetsky-Shapiro, G. \harvardand\ Smyth, P.  \harvardyearleft
  1996\harvardyearright , From data mining to knowledge discovery: An overview,
  {\em in} U.~Fayyad, G.~Piatetsky-Shapiro, P.~Smyth \harvardand\
  R.~Uthurusamy, eds, `Advances in Knowledge Discovery and Data Mining', AAAI
  Press/ MIT Press, pp.~1--34.

\harvarditem{Snodgrass}{1987}{Snodgrass87}
Snodgrass, R.  \harvardyearleft 1987\harvardyearright , `The temporal query
  language tquel', {\em ACM Transactions on Database Systems} {\bf
  12}(2),~247--298.

\end{thebibliography}


\end{document}

