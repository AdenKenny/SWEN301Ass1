\documentclass{CRPITStyle} 
\usepackage{harvard}  
\pagestyle{empty}
\thispagestyle{empty}
\hyphenation{roddick}

\begin{document}

	\title {Software Development Life Cycles: History and Future}
	\author {Aden Kenny}
	\maketitle


	\begin {abstract} 
		This paper discusses the importance of the use of process models in the software development lifecycle, provides a
		review of the process models used so far, compares and contrasts five different process models and then attempts to
		discuss the future of process models.
	\end {abstract}

	
	\section {Introduction} 
		\textit{The Software Crisis} was a time period in the infancy of software development where it was found to be difficult 
		to write software that was useful and efficient on the hardware of the time. This led to a large quantity of poor quality
		software being created that did not meet the requirements or was not even delivered at all.\\
		~\\
		Naur and Randell (1969) state that there were problems of "achieving sufficient reliability" and difficulties of meeting time
		requirements and specifications on large scale projects amongst other problems that were discussed at the \textit{NATO
		Software Engineering Conference, 1968}.\\
		~\\
		As a result of this conference where major problems with the state of software development were brought up, software
		development process models started to become far more popular as they were seen as a major part of the solution to the
		\textit{Software Crisis}.
	
	\bibliographystyle{agsm}   

\begin{thebibliography}{xx}

	\harvarditem{[Naur and Randell]}
  	{1993}{AIS93}
	Naur, P. \harvardand\ Randell, B.  \harvardyearleft
  	1969\harvardyearright , Software Engineering - Report a conference sponsored by the NATO Science Committee.


\end{thebibliography}
		

\end{document}